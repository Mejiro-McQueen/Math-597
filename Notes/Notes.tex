\documentclass[10pt,a4paper]{article}
\usepackage[utf8]{inputenc}
\usepackage[T1]{fontenc}
\usepackage{amsmath}
\usepackage{amsfonts}
\usepackage{amssymb}
\usepackage{graphicx}
\usepackage{amsthm}
\usepackage{hyperref}

\newtheorem{thm}{Theorem}[section]
\newtheorem{lem}[thm]{Lemma}
\theoremstyle{definition}
\newtheorem{defn}{Definition}[section]

\newtheorem{note}{Note}[section]


\begin{document}
	\section{Useful lemmas and notes from relatively $r$-prime }
	\begin{lem}\label{alternating-sum-to-prod}
		\[\prod_{i=1}(1-x_i)=\sum_{R\subset Q}(-1)^{|R|}\prod_{x\in R}x\] where $Q=\mathcal{P}(x_1, x_2, \mathellipsis)$ 
	\end{lem}
	\begin{proof}
		Applying the distributive property on $\prod_{i=1}(1-x_i)$ two terms at a time will reveal a pattern we can use for the next step.\newline 
		\begin{align}
			\prod_{i=1}(1-x_i)&= (1-x_1)(1-x_2)\prod_{i=3}(1-x_i)\\
			&=1-(x_1+x_2)+(x_1x_2)\prod_{i=3}(1-x_1)\\
			&=(1-(x_1+x_2)+(x_1x_2))(1-x_3)\prod_{i=4}(1-x_i)\\
			&=1-(x_1+x_2+x_3)+(x_1x_2+x_1x_3+x_2x_3)-(x_1x_2x_3)\prod_{i=4}(1-x_i) \label{partial_prod}
		\end{align}
	Then line \ref{partial_prod} can be expressed as \[\prod_{i=1}(1-x_i) = 1+(-1)\sum_{0<i}x_i+\sum_{0<i<j}x_ix_j+(-1)\sum_{0<i<j<k}x_ix_jx_k+\mathellipsis\]
	
	Now let $Q$ be the power set of our variables $x_1,x_2,\mathellipsis$, which is as follows:\newline $Q=\{\{\},\{x_1\},\{x_2\},\{x_3\},\{x_1,x_2\},\{x_1,x_3\},\{x_2,x_3\},\mathellipsis\}$
	
	Finally, we can express our previous sum as described in the lemma: 
	\[\prod_{i=1}(1-x_i)=\sum_{R\subset Q}(-1)^{|R|}\prod_{x\in R}x\]   
	\end{proof}

	\begin{defn}
		The Riemann Zeta functions is defined by \[\zeta(s)=\sum_{j=1}\frac{1}{j^s}=\prod_{p \text{ prime}}(1-p^{-s})^{-1}\]. 
	\end{defn}	
	
	\begin{defn}
		The Mobius function is given by:
		$\mu(n)=\begin{cases} 
			1 & \text{if } n=\pm 1 \\
			0 & \text{if } n \text{ is not square free} \\
			(-1)^r & \text{if } n=p_1p_2\mathellipsis p_r
		\end{cases}
		$
		
	\end{defn}
	\begin{lem}
		$\sum_{j=1}\frac{\mu(j)}{j^s}=\prod(1-p^{-s})=\frac{1}{\zeta(s)}$
	\end{lem}
	\begin{proof}
		We can apply the previous lemma to show a relationship between the Mobius function and the inverse Riemann Zeta function.
		
		Applying the lemma, $\prod(1-p^{-s})=\sum_{R\subset Q}(-1)^{|R|}\prod_{x\in R}x$ where $Q=\mathcal{P}(p_1,p_2,\mathellipsis,p_k)$
		\newline 
		\newline
		
		For any term, we have $R=\{p_1,p_2,\mathellipsis, p_r\}$. Then $(-1)^{|R|} = 1$ whenever $r$ is even, and $-1$ whenever $r$ is odd.
		Notice that these two cases coincide exactly with the last case of the Mobius function.			
		Thus, we can replace the sign of the term with $\mu(j)$ where $j=\prod_{x\in R}x=p_1p_2\mathellipsis p_r$ and we can replace the product of the term, which is given by $(p_1p_2\mathellipsis p_r)^{-s}$, with $\frac{1}{j^{s}}$. 
		
		Showing the lemma.
	\end{proof}

\begin{defn}[Floor]
	$\lfloor x \rfloor = n$ if and only if $n\leq x < n+1$
\end{defn}

\begin{lem}(Floor Annihilators)
	Given a positive rational $\frac{a}{b}$, $\lfloor \frac{a}{b}\rfloor = 0 $ if and only if $a<b$. 
\end{lem}
\begin{proof}
	Suppose $\lfloor \frac{a}{b} \rfloor = 0$. Then $n=0$ and from the definition of floor, $0\leq \frac{a}{b} < 0+1$. Then solving for $a$, $0\leq a < b$.
	
	Next, suppose $a<b$, then working in reverse, $\frac{a}{b}<1$  and so $0\leq \frac{a}{b} < 1$, which implies $\lfloor \frac{a}{b} \rfloor = 0$. 
\end{proof}

\begin{lem}
	Given a positive rational $\frac{n}{j^r}$, $\lfloor\frac{n}{j^r}\rfloor = 0 $ whenever $j>\lfloor \sqrt[r]{n}\rfloor$
\end{lem}
\begin{proof}
	Applying the previous lemma, $n<j^r$ and solving for $j$, $\sqrt[r]{n}<j$. We need $j$ to be an integer, so, $\lfloor\sqrt[r]{n}\rfloor\leq \sqrt[r]{n} < j$.
\end{proof}

\begin{defn}
	Given a fixed integer $r\geq 1$, the integers $m_1,m_2,\mathellipsis,m_k$ are relatively $r$-prime if and only if they have no common factor of the form $n^r$ for $n>1$. 
\end{defn}
\begin{lem}
	An ordered $k$-tuple of integers are relatively $r$-prime if and only if there exists no prime $p$ such that $p^r$ divides each $k$ integer. 
\end{lem}
\begin{proof}
	Suppose $M=(m_1,m_2,\mathellipsis,m_k)$ is a $k$-tuple that is relatively $r$-prime. Then for a fixed $r$, and any integer $n$, we have $n^r\nmid m_i$ for any $m_i\in M$. Next, $n$ is an integer and has a prime factorization, $p_1^{\alpha_1}p_2^{\alpha_2}p_3^{\alpha_3}\mathellipsis p_n^{\alpha_n}$. Thus, $n^r=(p_1^{\alpha_1})^r(p_2^{\alpha_2})^r(p_3^{\alpha_3})^r\nmid m_i$ for all $m_i\in M$. 
	
	Suppose there exists no prime $p$ such that $p^r$ divides each $k$ integer in a $k$-tuple called $M$. By prime factorization, we can construct any arbitrary integer $n$. Then, $(p_1^{\alpha_1})^r(p_2^{\alpha_2})^r(p_3^{\alpha_3})^r=(p_1^{\alpha_1}p_2^{\alpha_2}p_3^{\alpha_3})^r=n^r\nmid m_i$ for all $m_i\in M$.
\end{proof}

\begin{lem}
	$A^k-B^k=(A-B)(A^{k-1}+A^{k-1}B+A^{k-1}B^2\mathellipsis+B^{k-1})$
\end{lem}
\begin{proof}
	Applying the distributive property will show that only the first and last terms do not cancel.
\end{proof}

\begin{note}
	How many unique ordered $k$-tuples from elements in $S$ be made? 
	
	Using the fundamental counting principal: We have $|S|$ choices for each of the $k$ tuple elements, for a total of $|S|^k$ tuples. 
	
	How many ordered $k$-tuples can we make of positive integers less than or equal to $n$?
	
	$S=\{1,2,3,\mathellipsis,n\}$ implies $n^k$.
\end{note}

\begin{note}
	How many positive integers less than $n$ are divisible by $m$? 
	
	First, we will observe the following pattern and build upon it: 
	Suppose $n=100$ and $m=25$.
	Then $S=\{25,50,75,100\}$ are all divisible by $m$ and less than or equal to $n$. Also, $\frac{n}{m}=\frac{100}{25} = 4 = |S|$.
	
	However,
	Suppose $n=101$ and $m=25$.
	Then $S=\{25,50,75,100\}$ are all divisible by $m$ and less than or equal to $n$. Also, $\frac{n}{m}=\frac{101}{25} = 4.04.$
	
	The pattern does not hold unless we truncate the fractional part, which we can do by applying the floor function. 
	
	Let's start over with the inclusion of the floor function in order to find a more consistent pattern.
	Suppose $n=100$ and $m=25$.
	Then $S=\{25,50,75,100\}$ are all divisible by $m$ and less than or equal to $n$. Also, $\lfloor\frac{n}{m}\rfloor=\lfloor\frac{100}{25}\rfloor = \lfloor 4 \rfloor = 4 = |S|$.
	
	Suppose $n=101$ and $m=25$.
	Then $S=\{25,50,75,100\}$ are all divisible by $m$ and less than or equal to $n$. Also, $\lfloor\frac{n}{m}\rfloor=\lfloor\frac{101}{25}\rfloor = \lfloor 4.03 \rfloor = 4 = |S|$.
	
	Alternatively, and purely for demonstrative purposes, suppose we want to count the number of positive integers less than $n=162$ that are divisible by $m=25$. Construct the infinite arithemtic sequence $S=\{25,50,75,100,125,150,175,\mathellipsis\}$. Clearly, all elements divisible by $25$ are contained in $S$, we now need to filter elements that we are not interested in, $S^\prime=\{25,50,75,100,125,150\}$ with $|S|=6$. However, this has the same meaning as $m=25$ divides $n=162$ evenly $|S|=6$ times (repeated addition is multiplication and the inverse of multiplication is division).
	
	As a final note that will be useful later on, if $m<n$ then we have $\lfloor \frac{m}{n} \rfloor = 0 $ since $\frac{m}{n}<1$.
\end{note}

\begin{defn}
	Let $q(n)$ be the number of ordered $k$-tuples less than $n$ that are relatively $r$-prime.
\end{defn}

\begin{note}
	Inclusion Exclusion Principle
	
	We can use the inclusion exclusion principle to develop a more concrete idea of the actual value of $q(n)$ that we can later expand on. 
	
	We won't be dealing with sets directly, but instead be using their cardinality. 
	
	Fix $n\in \mathbb{N}$.
	
	Let $a=n^k$ be the cardinality of the set of all $k$-tuples of elements less than $n$, that is, the cardinality of our universal set. 
	
	Then let $b$ be the cardinality of the set of all $k$-tuples of elements less than $n$ that are not $r$-prime. 
	
	How to find such a thing?
	If a number is not $r$-prime, then it is divisible by $p^r$ for all primes $p$. 
	
	How many positive integers are less than $n$ and divisible by $p^r$? By a previous result: $\lfloor \frac{n}{P^r}\rfloor$. 
	
	If an ordered $k$-tuple is not $r$-prime, then at least one of the tuple elements must be divisible by $p^r$. 
	
	Similar to a previous result, we then have $k$ elements to choose in our ordered $k$-tuple with $\lfloor \frac{n}{p^r}\rfloor$ elements to choose from. Thus we have $\lfloor \frac{n}{p^r}\rfloor^k$ ordered $k$-tuples that are not $r$-prime. 
	
	Up to this point, our prime $p$ has been fixed. Now, we must index our result across all primes and count the results.
	
	Thus we have the expression $\sum_{p_{1}^r}\lfloor \frac{n}{p_{1}^r}\rfloor^k$
	
	Notice that since $n$ is fixed, at some point $n<p_i$, and the sum will converge to 0 forever. 
	
	For the remaining terms we can continue the pattern required by the inclusion exclusion principle, for example, the next term $c$ would add the cardinality of the set of all ordered $k$-tuples with elements less than $n$ that are not $r$-prime, this time indexing over the product of the primes from the previous terms as well. 
	
	This satisfies the intersection of sets required by the inclusion exclusion principle.
	
	Thus, by the inclusion exclusion principle:
	
	$q(n) = n^k-\sum_{p^1}\lfloor \frac{n}{p_1^r}\rfloor^k + \sum_{(p_1 < p_2)}\lfloor \frac{n}{(p_1 p_2)^r}\rfloor^k - \sum_{(p_1 < p_2 < p_3)}\lfloor \frac{n}{(p_1 p_2 p_3)^r}\rfloor^k +\mathellipsis$
	
	TODO: There might be a way to compact this. 
\end{note}

\begin{note}
	We can perform the multiple estimate substitutions on $q(n)$ to yield $\frac{1}{\zeta(rk)}$.\newline
	
	Given \[q(n)=\sum_{j=1}^\infty = \mu(j)\Bigl\lfloor \frac{n}{j^r} \Bigr\rfloor^k\] we want to find an estimate that does not contain the floor function. 
	
	Using the definition of floor, we find the relationship $0\leq x - \lfloor x \rfloor < 1$.
	This suggests that $x$ and $\lfloor x \rfloor$ are relatively close enough that we can perform a substitution if we account for a growth estimate. 
	
	Then using a previous result, 
	\begin{align*}
		\Bigl\lfloor \frac{n}{j^r}\Bigr\rfloor^k -\left(\frac{n}{j^r}\right)^k =\left(\Bigl\lfloor \frac{n}{j^r}\Bigr\rfloor^k- \left(\frac{n}{j^r}\right)^k\right)\left(\Bigl\lfloor\frac{n}{j^r}\Bigr\rfloor^{k-1}+\Bigl\lfloor\frac{n}{j^r}\Bigr\rfloor^{k-2}\left(\frac{n}{j^r}\right)+\mathellipsis+\left(\frac{n}{j^r}\right)^{k-1}\right)
	\end{align*}
	Now to find the growth estimate.
		\begin{align*}
		O\left(\Bigl\lfloor \frac{n}{j^r}\Bigr\rfloor^k -\left(\frac{n}{j^r}\right)^k \right)=O\left(\Bigl\lfloor \frac{n}{j^r}\Bigr\rfloor^k- \left(\frac{n}{j^r}\right)^k\right)O\left(\Bigl\lfloor\frac{n}{j^r}\Bigr\rfloor^{k-1}+\Bigl\lfloor\frac{n}{j^r}\Bigr\rfloor^{k-2}\left(\frac{n}{j^r}\right)+\mathellipsis+\left(\frac{n}{j^r}\right)^{k-1}\right)
	\end{align*}
	As stated earlier, the first product $\left(\Bigl\lfloor \frac{n}{j^r}\Bigr\rfloor^k- \left(\frac{n}{j^r}\right)^k\right)$, this term is at most 1, so it can be treated as a constant. For the right most product, $\lfloor x \rfloor^mx^n\leq x^{m+n}$, and so the fastest growing term is  $\left(\frac{n}{j^r}\right)^{k-1}$. Thus  $\Bigl\lfloor \frac{n}{j^r}\Bigr\rfloor^k -\left(\frac{n}{j^r}\right)^k=O\left(\left(\frac{n}{j^r}\right)^{k-1}\right)$. 
	
	\begin{align}
	 q(n)&=\sum_{j=1}^{\lfloor\sqrt[r]{n}\rfloor}\mu(j)\left(\frac{n}{j^r}\right)^k+\sum_{j=1}^{\lfloor\sqrt[r]{n}\rfloor}\mu(j)O\left(\left(\frac{n}{j^r}\right)^{k-1}\right)\\
	 &=\sum_{j=1}^{\lfloor\sqrt[r]{n}\rfloor}\mu(j)\left(\frac{n}{j^r}\right)^k+O\left(\sum_{j=1}^{\lfloor\sqrt[r]{n}\rfloor}\mu(j)\left(\frac{n}{j^r}\right)^{k-1}\right)  \label{two_estimates}
	\end{align}
	
	We can now find an estimate for the new estimate that was induced by our first estimate. 
	
	\begin{align}
	O\left(\sum_{j=1}^{\lfloor\sqrt[r]{n}\rfloor}\mu(j)\left(\frac{n}{j^r}\right)^{k-1}\right) &= 
	O\left(n^{k-1}\sum_{j=1}^{\lfloor\sqrt[r]{n}\rfloor}\mu(j)\left(\frac{1}{j^{r(k-1)}}\right)\right)\\
	&= 
	O\left(n^{k-1}\right)O\left(\sum_{j=1}^{\lfloor\sqrt[r]{n}\rfloor}\mu(j)\left(\frac{1}{j^{r(k-1)}}\right)\right)\\
	&=O\left(n^{k-1}\right)O\left(\sum_{j=1}^{\lfloor\sqrt[r]{n}\rfloor}\mu(j)\right)O\left(\sum_{j=1}^{\lfloor\sqrt[r]{n}\rfloor}\left(\frac{1}{j^{r(k-1)}}\right)\right)\\
	&=O\left(n^{k-1}\right)O\left(\sum_{j=1}^{\lfloor\sqrt[r]{n}\rfloor}\left(\frac{1}{j^{r(k-1)}}\right)\right) \label{estimate_second}
	\end{align}

	Focusing on the rightmost estimate, and applying the Riemann integral,
	\begin{align}
		O\left(\sum_{j=1}^{\lfloor\sqrt[r]{n}\rfloor}\left(\frac{1}{j^{r(k-1)}}\right)\right)&\leq O\left(1+\int_{1}^{\sqrt[r]{n}}\frac{dx}{x^{r(k-1)}}\right) \\
		&= \begin{cases}
			O(\ln n) \text{ if r=1 and k=2}\\
			O(n^{\frac{1}{r}}) \text{ if } r\leq 2 \text{ and } k=1\\
			O(1) \text{ otherwise}
		\end{cases}
	\end{align} 
	Then substituting this estimate into line \ref{estimate_second}, 
	
	\begin{align}
	O\left(n^{k-1}\right)O\left(\sum_{j=1}^{\lfloor\sqrt[r]{n}\rfloor}\left(\frac{1}{j^{r(k-1)}}\right)\right)&=O(n^{k-1})\cdot\begin{cases}
		O(\ln n) \text{ if r=1 and k=2}\\
		O(n^{\frac{1}{r}}) \text{ if } r\leq 2 \text{ and } k=1\\
		O(1) \text{ otherwise}
	\end{cases}\\
	&=\begin{cases}
	O(n\ln n) \text{ if r=1 and k=2}\\
	O(n^{\frac{1}{r}}) \text{ if } r\leq 2 \text{ and } k=1\\
	O(n^{k-1}) \text{ otherwise}
\end{cases}
	\end{align}
	Now we want to find and estimate to the left most sum on line \ref{two_estimates}. By applying a previous lemma,
	\begin{align}		
		\sum_{j=1}^{\lfloor\sqrt[r]{n}\rfloor}\mu(j)\left(\frac{n}{j^r}\right)^k &= \frac{n^k}{\zeta(rk)} - n^k\sum_{\sqrt[r]{n}+1}^\infty \frac{\mu(j)}{j^{rk}}
	\end{align}
	Now we can estimate the rightmost sum by applying a Riemann integral. 
	\begin{align}
		O\left(n^k\sum_{\sqrt[r]{n}+1}^\infty \frac{\mu(j)}{j^{rk}}\right) &= O(n^k)O\left(\sum_{\sqrt[r]{n}+1}^\infty \frac{\mu(j)}{j^{rk}}\right)\\
		&\leq O(n^k)O\left(\int_{\sqrt[r]{n}}^{\infty}\frac{\mu(j)}{j^{rk}}\right)\\
		&=O(n^k)O\left(n^{{1}/(r-k)}\right)\\
		&=O\left(n^{{1}/r}\right)
	\end{align} 

	Putting all the estimates together, 
	\begin{align}
		q(n) &= \frac{n^k}{\zeta(rk)}+O\left(n^{{1}/r}\right)+ \begin{cases}
			O(n\ln n) \text{ if r=1 and k=2}\\
			O(n^{\frac{1}{r}}) \text{ if } r\leq 2 \text{ and } k=1\\
			O(n^{k-1}) \text{ otherwise}\end{cases}\\
		&= \frac{n^k}{\zeta(rk)}+ \begin{cases}
			O(n\ln n) \text{ if r=1 and k=2}\\
			O(n^{\frac{1}{r}}) \text{ if } r\leq 2 \text{ and } k=1\\
			O(n^{k-1}) \text{ otherwise}\end{cases}
	\end{align}
	Finally, $\lim\limits_{n\to\infty} n^k = \infty$ and $\lim\limits_{n\to\infty} q(n)=\infty$, and so we can apply L`Hoptial's rule:
	
	\begin{align*}
		\lim\limits_{n\to\infty} \frac{q(n)}{n^k} = \frac{1}{\zeta(rk)}
	\end{align*}
\end{note}
\section{B-Friable}
\begin{note}
	Fix $m>2,n\in \mathbb{Z}^+$.
	For each $1\leq i \leq m$, $1\leq j\leq n$, randomly choose $r\in [1,N]$ to form a set of ordered pairs $(r_{ij})$.
\end{note}

\begin{defn}
	Let $B$ be a fixed positive integer.
	
	A positive integer $x$ is B-friable (B-smooth) if and only if $x$ has no prime divisor greater than $B$.
	
	Equivalently:
	A positive integer $x$ is B-Friable (B-smooth) if and only if none of its prime factors are greater than B. 
\end{defn}

We now want to count the number of $B$-friable random and independently chosen $(r_{ij})$ pairs. 
\begin{defn}
	Let $B$ be a fixed integer.
	
	Let $X_l=\{p_1,p_2,\mathellipsis,p_l\}$ be a subset of all primes larger than $B$. 
	
	Define $T(l,N)$ to be the number of ordered pairs $(r_{ij})$ such that $\gcd(\prod_{j=1}^nr_{1j},\mathellipsis,\prod_{j=1}^{n}r_{mj})$ is coprime to all $p\in X_l$.  
\end{defn}

\begin{thm}
	content...
\end{thm}
\begin{proof}
	\href{https://en.wikipedia.org/wiki/Inclusion%E2%80%93exclusion_principle#A_generalization}{Reference INC/EXC Generalization}\newline
		
	$T(l,N)=|\{(r_{ij}) : \gcd(\prod_{j=1}^nr_{1j},\mathellipsis,\prod_{j=1}^{n}r_{mj})  \text{ is coprime to }p\in X_l \}|$
	
	Using a generalized form of the inclusion exclsuion principle, we can count the nuber of ordered pairs that are not in a subset containing elements that are not coprime to some $p\in X_l$.
	
	Then
	$|\{(r_{ij}) :|\gcd(\prod_{j=1}^nr_{1j},\mathellipsis,\prod_{j=1}^{n}r_{mj}) \text{ is coprime to } p\in X_l\}| = \newline \sum_{P\subset X_l}{(-1)^{|P|}}|\{(r_{ij}:\prod_{p\in P} p\mid \gcd(\prod_{j=1}^nr_{1j},\mathellipsis,\prod_{j=1}^{n}r_{mj})\} = \sum_{P\subset X_l}{(-1)^{|P|}}|\{(r_{ij}:\prod_{p\in P} p\mid \prod_{j=1}^nr_{1j}\}|^m$
	
	Next, we need to apply the same inclusion exclusion principle form.
	
	$|\{(r_{ij}:\prod_{p\in P} p\mid \prod_{j=1}^nr_{1j}\}|^m=\sum_{q\subset P}(-1)^{|Q|}|\{(r_{1j}):p\nmid \prod_{j=1}^nr_{1j,\forall p\in Q}\}=\sum_{Q\subset P}(-1)^{|Q|}(\sum_{R\subset Q}(-1)^{|R|}\lfloor\frac{N}{\prod_{p\in R p}}\rfloor)^n$
	
	Finally, we have $T(l,N)=\sum_{P\subset X_l}{(-1)^{|P|}}(\sum_{Q\subset P}(-1)^{|Q|}(\sum_{R\subset Q}(-1)^{|R|}\lfloor\frac{N}{\prod_{p\in R p}}\rfloor)^n)^m$
\end{proof}

\begin{lem}
	$\frac{1}{N} \lfloor \frac{N}{\prod_{p\in R p}}\rfloor = \frac{1}{\prod_{p\in P}p}+O(\frac{1}{N})$
\end{lem}
\begin{proof}
	Similar to the previous problem:
	\begin{align*}
		0\leq \frac{N}{\prod_{p\in R p}}-\lfloor\frac{N}{\prod_{p\in R p}}\rfloor<1\\
		\implies \frac{0}{N}\leq \frac{1}{\prod_{p\in R p}}-\frac{1}{N}\lfloor\frac{N}{\prod_{p\in R p}}\rfloor <\frac{1}{N}\\ 
	\end{align*}
	Thus, the error of substituting $\frac{1}{\prod_{p\in R p}}$ is at most $O(\frac{1}{N})$
\end{proof}

\begin{thm}
	$\lim\limits_{N\to\infty}\frac{T(l,N)}{N^{nm}}=\prod_{i=1}^{l}(1-(1-(1-\frac{1}{p^i})^n)^m)$
\end{thm}
\begin{proof}
	$\lim\limits_{N\to\infty}\frac{T(l,N)}{N^{nm}}=\lim\limits_{N\to\infty}\frac{1}{N^{nm}}\sum_{P\subset X_l}{(-1)^{|P|}}(\sum_{Q\subset P}(-1)^{|Q|}(\sum_{R\subset Q}(-1)^{|R|}\lfloor\frac{N}{\prod_{p\in R p}}\rfloor)^n)^m=\lim\limits_{N\to\infty}\sum_{P\subset X_l}{(-1)^{|P|}}(\sum_{Q\subset P}(-1)^{|Q|}(\sum_{R\subset Q}(-1)^{|R|}\frac{1}{\prod_{p\in R p}})^n)^m+O(\frac{1}{N^{nm}})$
	
	Using multiple applications of a lemma (\ref{alternating-sum-to-prod}):\begin{align*}
	\lim\limits_{N\to\infty}\sum_{P\subset X_l}{(-1)^{|P|}}(\sum_{Q\subset P}(-1)^{|Q|}(\sum_{R\subset Q}(-1)^{|R|}\frac{1}{\prod_{p\in R p}})^n)^m+\lim\limits_{N\to\infty}O(\frac{1}{N^{nm}})=\\
	\lim\limits_{N\to\infty} \prod_{i=1}^{l}(1-(1-(1-\frac{1}{p_i})^n)^m)+\lim\limits_{N\to\infty}O(\frac{1}{N^{mn}}) 
	\end{align*}
	
	Taking the limit yields the hypothesis.
\end{proof}
This gives the probability that the gcd of products of integers is not divisible by the first $l$ primes greater than $B$.

Now we want to find the same probability, this time for an infinite number (all) of primes greater than $B$, thus giving the probability that the GCD is $B$-Friable. 

However, do we take $l\to \inf$ before or after $n\to\infty$? Or does order not matter? The final algebra suggests that only the error depends on $N$, so we don't have to worry about it? TODO: Ask professor.

\begin{thm}
	$\lim\limits_{l\to\infty}\lim\limits_{N\to\infty}\frac{T(l,N)}{N^{nm}}=\prod_{i=1}^{l}(1-(1-(1-\frac{1}{p^i})^n)^m)$
\end{thm}

Now we want to find upper and lower bounds. 

\begin{thm}
	Let $\hat{n}=\max\{n,B\}$, $r=\lfloor n^{\frac{m}{m-1}}+1\rfloor$, and $\hat{r}=\max\{\hat{n},r\}$.
	
	Then the probability that $\gcd(\prod_{j=1}^n r_{1j},\mathellipsis,\prod_{j=1}^{n}r_{mj})$ is $B$-friable is bounded by $\frac{1}{\zeta(m)}\prod_{p\leq B}(1-\frac{1}{p^m})^{-1}$ and is lower bounded by $\prod_{B<p\leq\hat{n}}[1-\{1-(1-\frac{1}{p})^n\}^m]\prod_{\hat{n}<p\leq\hat{r}}\{{1-(\frac{n}{p}^m)}\}\frac{1}{\zeta(s)}$. Where the first product term is equal to 1 if $B=\hat{n}$, and the second product term is equal to 1 if $\hat{n}=\hat{r}$.
	\end{thm} 
	\begin{proof}
		$(1-\frac{1}{p})^n$ decreases as $n$ increases so  $\prod_{p>B}[1-\{1-(1-\frac{1}{p})^n\}^m]$ decreases as $n$ increases. This gives us the inequality $\prod_{B<p\leq\hat{n}}[1-\{1-(1-\frac{1}{p})^n\}^m] \leq \prod_{p>B}(1-\frac{1}{p^m})=\frac{1}{\zeta(m)}\prod_{p\leq B}(1-\frac{1}{p^m})^{-1}$.
		
		Where the rightmost equality is derived using the Riemann Zeta function definition. Thus we have found  the upper bound.
		
		Next, using Bernoulli's inequality, $(1-\frac{n}{p})^n \leq 1- \frac{n}{p}$. If $p>n$, then the entire expression is less than 1. Thus if we set $p>\max\{n,B\}=\hat{n}$ we can always satisfy this condition. 
		
		Thus we have $\prod_{B<p\leq\hat{n}}[1-\{1-(1-\frac{1}{p})^n\}^m] \geq\prod_{B<p\leq \hat{n}}[1-\{1-(1-\frac{1}{p})^n\}^m]\cdot\prod_{p>\hat{n}}\{1-(\frac{n}{p})^m\}$ 
		
		TODO: STUFF FROM SCRATCH PAD
		
		Next, $\prod_{\hat{n}<p\leq\hat{r}}\{1-\frac{n}{p}^m\} < 1$ since $p>n$ and $\prod_{p>\hat{r}}(1-\frac{1}{p^s})<1$. 
		
		Thus, 
	 $\prod_{B<p\leq\hat{n}}[1-\{1-(1-\frac{1}{p})^n\}^m] \geq\prod_{B<p\leq \hat{n}}[1-\{1-(1-\frac{1}{p})^n\}^m]\cdot\prod_{p>\hat{n}}\{1-(\frac{n}{p})^m\}\cdot\prod_{p>\hat{r}}(1-\frac{1}{p^s})>\prod_{p>\hat{n}}\{1-(\frac{n}{p})^m\}\cdot\prod_{p>\hat{r}}\frac{1}{\zeta(s)}$, which covers all cases for $p>B$. 
	 
	 \begin{thm}
	 	Remark 2.4 here, blocked by derivation of $s$
	 \end{thm} 
	 
	\end{proof}
	\subsection{Scratchpad}
	"We can easily check that $\frac{n^m}{p^m}\leq \frac{1}{p^s}$" 
	
	Can't seem to figure it out.
	Not even sure why this is needed.

	How do I derive $s$?
	Solving with logs gets close, but can't deal with the denominator.

	\begin{thm}
		Let $s=m(1-\log_{\hat{r}}n)$. Then, $s>1$ whenever $0\leq n\leq\hat{r}=\max\{n,B,r\}$. 
	\end{thm}
	\begin{proof}
		$\log_{\hat{r}}n<1$ whenever $0\leq n\leq\hat{r}=\max\{n,B,r\}$.
		and for positive integer $m\geq 2$,
		then $m(1-\log_{\hat{r}}n)>1$. 
	\end{proof}

	\begin{thm}
		$\hat{r}\geq r > n^{{\frac{m}{m-1}}}$
	\end{thm}
	\begin{proof}
		Let $r=\lfloor n^{\frac{m}{m-1}}+1\rfloor$
		Then using the definition of floor, $r\geq n^{\frac{m}{m-1}}\geq r-1$
		
		Next, $\max\{n,B,r\} = \hat{r} \geq r \geq n^{\frac{m}{m-1}}\geq r-1$. 
	\end{proof}


	\section{Generalization to $k$-gcd}
	
	\begin{defn}
		Let $\gcd_k(x_1,x_2,\mathellipsis,x_n)$ be the largest integer whose $k^{th}$ power divides $x_1, x_2,\mathellipsis, x_n$.
		
		When $k=1,\gcd_k(x_1,x_2,\mathellipsis,x_n)=\gcd(x_1,x_2,\mathellipsis,x_n)$
		
		When $\gcd_k(x_1,x_2,\mathellipsis,x_n)=1$ we say that $ x_1,x_2,\mathellipsis,x_n$ are relatively $k$-prime.
	\end{defn}
	Example: 
	\begin{align*}
		\gcd{_1}(128,192,1280)&=64\\
		\gcd{_2}(128,192,1280)&=8\\
		\gcd{_3}(128,192,1280)&=4\\
		\gcd{_4}(128,192,1280)&=2\\
		\gcd{_5}(128,192,1280)&=2\\
		\gcd{_6}(128,192,1280)&=2\\
		\gcd{_7}(128,192,1280)&=2\\
		\gcd{_8}(128,192,1280)&=1\\
	\end{align*}
	

	\begin{thm}
		$\prod_{p\in P} \mid \gcd_k(\prod_{j=1}^{n}r_{1j},\mathellipsis,\prod_{j=1}^n r_{mj})$ if and only if $\prod_{p\in P} p^k \mid \prod_{j}r_{ij}$ for any $i$.
	\end{thm}

	\begin{proof}
		 Suppose $\prod_{p\in P} \mid \gcd_k(\prod_{j=1}^{n}r_{1j},\mathellipsis,\prod_{j=1}^n r_{mj})=a$.
		 
		 Then. $p|a \iff p^k|a^k \iff p^k | \prod_j r_{ij}$ for some $i$. 
	\end{proof}

	\begin{thm}
		Let $p_1, p_2,\mathellipsis$ be the prime numbers larger than $B$ in increasing order. 
		
		$\lim\limits_{N\to\infty} \frac{T_k(l,n)}{N^{mn}}=\prod_{i=1}^l[1-\{1-(\frac{1}{p_i})^n(1+\frac{nH_1}{p_i})+\mathellipsis+\frac{nH_{k-1}}{p_i^{k-1}})\}]$
	\end{thm}

	\begin{proof}
		  Let $X_l=\{p_1,\mathellipsis,p_l\}$ and $1\leq r_{ij}\leq N$
		  
		  Using the inclusion exclusion theorem, $\frac{T_k(l,n)}{N^{nm}}=\sum_{P\subset X_l}(-1)^{|P|}(\sum_{Q\subset P}\Pr[p^k\nmid \prod_{j=1}^{n}r_{1j} \; \forall p\in Q ])^m$
		  
		  Let $p^a \mid\mid x$ denote $p^a\mid x$ and $p^{a+1}\nmid x$ and let $a_p,j\in Q$ and $1\leq j \leq n$.
		  
		  Using a result from combinatorics: 
		  \url{https://en.wikipedia.org/wiki/Stars_and_bars_(combinatorics)}
		  \url{https://en.wikipedia.org/wiki/Composition_(combinatorics)}
		  
		  The number of $n$-tuples of positive integers which satisfy $a_{p,1}+\mathellipsis+a_{p,n}=i$ is given by $_nH_i = (\binom{n+i-1}{i})$
		  
		  Regarding $||$, suppose a tuple satisfies $a_{p,1}+\mathellipsis+a_{p,n}<k$ with $a_{p,1}+\mathellipsis+a_{p,n}+1=k$, then $||$ guarantees $p^k \nmid x$. 
		  
		  We will use these two concepts to count every composition $a_{p,1}+\mathellipsis+a_{p,n}<k$ for $p^{a_{p,j}}$ that divides the quantity.
		  
		  Thus $\Pr[p^k\nmid \prod_{j=1}^{n}r_{1j}\:\forall p\in Q] = \sum_{a_{p,1}+\mathellipsis+a_{p,n}<l}\Pr[p^{a_{p,j}} \mid\mid r_{1j} \forall p,j]=\sum_{a_{p,1}+\mathellipsis+a_{p,n}<k}\prod_{j=1}^{n}\Pr[p^{a_{p,j}} \mid\mid r_{1j} \forall p\in Q]$, where the last term utilizes the product rule for probability. This allows us to count the innermost quantity. 
		  
		  Using the first form of the inclusion exclusion principle discussed earlier, $|\{(r_{1j}): p^{a_{{p,j}}}\mid\mid r_{1j} \: \forall p\in Q\}| = \lfloor\frac{N}{\prod_{p\in Q}p^{a_{p,j}}}\rfloor-\sum_{p\in Q}\lfloor \frac{N}{q\prod_{p\in Q}p^{a_{p,j}}}\rfloor+\mathellipsis+(-1)^{|Q|}\lfloor\frac{N}{\prod_{p\in Q}p^{a_{p,j}+1}}\rfloor$
		  
		  Next, we form an estimate of the quantity by taking the difference of the last two terms. 
		  
		  $\lfloor \frac{N}{\prod {p\in Q} p^{a_{p,j}}} \rfloor - \lfloor \frac{N}{\prod {p\in Q} p^{a_{p,j}+1}} \rfloor =  \frac{N}{\prod {p\in Q} p^{a_{p,j}}} + O(1) -  \frac{N}{\prod {p\in Q} p^{a_{p,j}+1}} + O(1) = N\prod_ {p\in Q}[ \frac{1}{ p^{a_{p,j}}}  -  \frac{1}{ p^{a_{p,j}+1}} ]+ \frac{2}{N}O(1)=N\prod_ {p\in Q}[ \frac{1}{ p^{a_{p,j}}}  -  \frac{1}{ p^{a_{p,j}+1}} ]+ O(1)$ 
		  
		  Thus we have $\Pr[p^k\nmid \prod_{j=1}^{n}r_{1j}\:\forall p\in Q] = \prod_{p\in Q} (\sum_{a_{p,1}+\mathellipsis+a_{p,n}<k}\prod_{j=1}^{n}\frac{p-1}{p^{a_{p,j}+1}})+O(\frac{1}{N}) = \prod_{p\in Q}[(1-\frac{1}{p})^n\sum_{a_{p,1}+\mathellipsis+a_{p,n}<k}\frac{1}{p^{a_{p,1}+\mathellipsis+a_{p,n}}}]+O(\frac{1}{N})$
		  
		  Finally, invoking an earlier result,  $\prod_{p\in Q}[(1-\frac{1}{p})^n\sum_{a_{p,1}+\mathellipsis+a_{p,n}<k}\frac{1}{p^{a_{p,1}+\mathellipsis+a_{p,n}}}]+O(\frac{1}{N})= \prod_{p\in Q}[(1-\frac{1}{p})^n(1+\frac{_nH_1}{p}+\mathellipsis+\frac{_nH_{k-1}}{p^{k-1}})]+O(\frac{1}{N})$
		  
		  Finally, substitute these results into the original expression and repeatedly apply (\ref{alternating-sum-to-prod}): to simplify as in an eariler result (TODO: ref), and let $N\to \infty$, and the error approaches 0. 
	\end{proof}

	\begin{proof}
		TODO: Still not convinced you need LDCT. Better ask professor about it. 
	\end{proof}

	TODO: What to do with last page? Not a lot of room for proofs, and the $s$ that I couldn't prove is invoked again.
	\section{Ideas?}
	\url{https://en.wikipedia.org/wiki/Smooth_number}
	
	What if I try to generalize for powersmooth / Ultrafriable? Maybe some other concept of smooth/rough?
\end{document}